\documentclass[book.tex]{subfiles}

\section{Preface}\label{preface}

Although book is not an ethnography, it has an ethnographic situation.
If it has a field site, it lies close to the places where the writing
was done -- in universities, on campuses, in classrooms and online
training courses (including MOOCs), and then amidst the books,
documents, websites, software manuals and documentation, and a rather
vast accumulation of scientific publications. It's a case of `dig where
you stand,' or `auto-archaeology.' \index{archaeology!auto-}

Readers familiar with textbooks in computer science and statistics can
detect the traces of this setting in various typographic conventions
drawn from the fields I write about. Important conventions include:

\begin{enumerate}
\def\labelenumi{\arabic{enumi}.}
\itemsep1pt\parskip0pt\parsep0pt
\item
  Typesetting the name of any code or devices that do machine learning
  and datasets on which machine learners operate in a \texttt{monospace}
  or terminal font: \texttt{machine\ learner} or \texttt{iris};
\item
  Presenting formulae, functions, and equations using the bristling
  indexicality of mathematical typography: \(\hat{\beta}\)
\end{enumerate}

I emulate the apparatus of science and engineering publication as an
experiment in \emph{in-situ} hybridization. Social science and
humanities researchers, even when they are observant participants in
their field sites, rarely experience a coincidence between their own
writing practices and that of the participants in the research site they
study. The object of study in this book, however, is a knowledge
practice that documents itself in code, equations, diagrams and
statements circulated in articles, books and various online formats
(blogs, wikis, software repositories). It is possible for a social
researcher to also adopt some of these practices.

I've been writing code for years \autocite{Mackenzie_2006}. Writing code
was nearly always something distant from writing about code since coding
was about software projects and writing was about thinking and
knowledge. I was slow to realise they are much entangled. Recent
developments in ways of analysing and publishing scientific data bring
coding and writing closer together. Implementing code can be done almost
in the same space, in the same screen or pane, as writing about code.
The mingling of coding and writing about code brings about sometimes
generative, sometimes frustrating, encounters with various scientific
knowledge (mathematics, statistics, computer science), with
infrastructures and devices on many scales (ranging across networks,
text editors, databases here and there, hardware and platforms of
various kinds, as well as interfaces) and many domains.

At many points in researching the book, I digressed a long way into
quite technical domains of statistical inference, probability theory,
linear algebra, dynamic models as well as database design and data
standards. In the interests of maintaining a strong feedback signal
running through the many propositions, formulations, diagrams,
equations, citations and images in this book, much of the code I've
written in implementing machine learning models or in reconstructing
certain data practices does not appear in this text, just as not all of
the words I've written in trying to construct arguments or think about
data practices has been included. Much has been cut away and left on the
ground (although the \texttt{git} repository of the book preserves many
traces of the writing and code; see
\url{https://github.com/datapractice/machinelearning}). As in the many
machine learning textbooks, recipe books, cookbooks, how-tos, tutorials
and manuals I have read, code, graphics and prose have been tidied here.
Many exploratory forays are lost and almost forgotten.

The several years I have spent doing and writing about data practice has
felt substantially different to any other project by virtue of the
hybridization between code in text, and text in code. Practically, this
is made possible by working on code and text within the same file, in
the same text editor. Switching between writing \texttt{R} and Python
code (about which I say more below) to retrieve data, to transform it,
to produce graphics, to construct models or some kind of graphic image,
and within the same file be writing academic prose, might be one way to
write about machine learning as a data practice.

The capacity to mingle text, code and images depends on an ensemble of
open source, often command-line software tools that differ somewhat from
the typical social scientist or humanities researchers' software toolkit
of word processor, bibliographic software, image editor and web browser.
In particular, I have relied on software packages in the \texttt{R}
programming language such as the `\texttt{knitr}'
\autocites{Xie_2013}{Xie_2012} and in \texttt{python,} the
\texttt{ipython} notebook environment \autocite{Perez_2007}. Both have
been developed by scientists and statisticians in the name of
`reproducible research.' \index{science!reproducible research} Many
examples of this form of writing can be found on the web: see
\href{http://nbviewer.ipython.org/}{IPython Notebook Viewer} for a
sample of these. These packages are designed to allow a combination of
code written in \texttt{R}, python or other programming languages,
scientific writing (including mathematical formula) and images to be
included, and importantly, executed together to produce a
document.\footnote{In order to do this, they typically combine some form
  of text formatting or `markup,' that ranges from very simple
  formatting conventions (for instance, the `Markdown' format used in
  this book is much less complicated than HTML, and uses markup
  conventions readable as plain text and modelled on email
  \autocite{Gruber_2004};) to the highly technical (LaTeX, the de-facto
  scientific publishing format or `document preparation system'
  \autocite{Lamport_1986} elements of which are also used here to convey
  mathematical expressions). They add to that blocks of code and inline
  code fragments that are executed as the text is formatted in order to
  produce results that are shown in the text or inserted as figures in
  the text.

  There are a few different ways of weaving together text, computation
  and images together. Each suffers from different limitations. In
  \texttt{ipython}, a scientific computing platform dating from 2005
  \autocite{Perez_2007} and used across a range of scientific settings,
  interactive visualization and plotting, as well as access to operating
  system functions are brought together in a \texttt{Python} programming
  environment. Especially in using the \texttt{ipython} notebook, where
  editing text and editing code is all done in the same window, and the
  results of changes to code can be seen immediately, practices of
  working with data can be directly woven together with writing about
  practice. By contrast, \texttt{knitr} generates documents by combining
  text passages and the results (graphs, calculations, tabulations of
  data) of code interleaved between the text into one output document.
  When \texttt{knitr} runs, it executes the code and inserts the results
  (calculations, text, images) in the flow of text.
  \index{R!packages!\textit{knitr}}

  Practically, this means that the text editor used to write code and
  text, remains somewhat separate from the software that executes the
  code. By contrast, \texttt{ipython} combines text and \texttt{Python}
  code more continuously, but at the cost of editing and writing code
  and text in a browser window. Most of the conveniences and affordances
  of text editing software is lost. While \texttt{ipython} focuses on
  interactive computation, \texttt{knitr} focuses on bringing together
  scientific document formatting and computation. Given that both can
  include code written in other languages (that is, \texttt{python} code
  can be processed by \texttt{knitr}, and \texttt{R} code executed in
  \texttt{ipython}), the differences are not crucially important. This
  whole book could have been written using just Python, since Python is
  a popular general purpose programming language, and many statistical,
  machine learning and data analysis libraries have been written for
  Python. I have used both, sometimes to highlight tensions between the
  somewhat more research-oriented \texttt{R} and the more practical
  applications typical of Python, and sometimes because code in one
  language is more easily understood than the other.
  \index{programming languages!R} \index{programming languages!Python}}
\index{programming languages!as mode of writing}

In making use of the equipment created by the people I study, I've
attempted to bring the writing of code and writing about code-like
operations into critical proximity. Does proximity or mixing of writing
code and writing words make a practical difference to an account of
practice? If recent theories of code and software as forms of speech,
expression or performative utterance \autocites{Cox_2012}{Coleman_2012},
or more generally praxiography as a reality-making descriptive practice
\autocite{Mol_2003} are right, it should. Weaving code through writing
in one domain of contemporary technical practice, machine learning,
might be one way of keeping multiple practices present, developing a
concrete sense of abstraction and allowing an affective expansion in
relation to machines.
