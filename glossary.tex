\longnewglossaryentry{vector}{name={vector}, description={ Three senses of the term are relevant: 1. A vector as an element of vector space; 2. A data structure -- a one dimensional array of elements; 3. A feeling in the sense used by A.N. Whitehead to describe the transfer from 'there' to 'here'.  }}
\newglossaryentry{archaeology}{name={archaeology}, description={Michel Foucault defines archaeology as a description that questions the already-said at the level of its existence. Alternately, archaeology describes discourses as practices specified in the element of the archive [@Foucault_1972, 131}}
\newglossaryentry{archive}{name={archive}, description={}}
\newglossaryentry{classifier}{name={classifier},plural={classifiers}, description={A machine learner that assigns instances to classes or categories.}}
\newglossaryentry{costfunction}{name={cost function}, description={A function that measures the difference between the output of the model (the prediction) and the known values.}}
\newglossaryentry{data strain}{name={data strain}, description={The term, which borrows from A.N. Whitehead's notion of strain refers to implicit forces or tensions in bodies of data that relate to the feeling of geometrically straight or flat loci.  }}
\newglossaryentry{decision boundary}{name={decision boundary}, description={A boundary or surface drawn in common vector space by a machine learning classifier to differentiate or separate and hence classify cases.}}
\newglossaryentry{deep learning}{name={deep learning}, description={a neural network comprising many layers commonly used for image recognition}}
\newglossaryentry{discourse}{name={discourse}, description={For Michel Foucault, a discourse holds together the contradictions in a group of statements generated by an enunciative function.}}
\newglossaryentry{enunciative function}{name={enunciative function}, description={For Michel Foucault, the specific relation of a statements to themselves, the relation to a subject, the correlate domain and their material existence in the form of reuses, replications and transcription together generate statements. In this book, the many predictions, inferences, plots, tabulations, numbers, scores, probabilities, classifications, software libraries and devices of machine learning comprise its enunciative function. }}
\newglossaryentry{enunciative modality}{name={enunciative modality}, description={For Michel Foucault, the sites, subject positions, forms of observing, describing, teaching, perceiving associated with statements.}}
\newglossaryentry{feature}{name={feature}, description={Also known in machine learning as variable, measurement, observation or attribute, a feature fills one dimension in the vector space inhabited by data. }}
\newglossaryentry{function}{name={function}, description={A mapping between two different sets of numbers. In machine learning, functions play very diverse roles, sometimes transforming data to generate feature or vector spaces, sometimes measuring cost or loss for particular models, and sometimes expressing forms such as curves and surfaces that traverse data. Across these different usages and domains, the operation of mapping or relation between sets of values such as $X$ and $Y$ can be seen.}}
\newglossaryentry{Naive Bayes}{name={Na\"{\i}ve Bayes}, description={A classifier that calculates the probability of an instance belonging to each class, and then assigns the instance to the most probable class.}}
\newglossaryentry{partial derivative}{name={partial derivative}, description={An operator from differential calculus that expresses the rate of change of one vvariable with respect to another.}}
\newglossaryentry{partial observer}{name={partial observer}, description=(Gilles Deleuze and Felix Guattari's concept of what a mathematical function does.}}
\newglossaryentry{perceptron}{name={Perceptron}, description={A machine learner developed in the 1950s by Frank Rosenblatt. It is modelled on a neurone that learns to classify the input data or what it 'perceives' by varying parameters or weights on the  sum of its inputs to produce values of either `1` or 0'}}
\newglossaryentry{positivity}{name={positivity}, description={Michel Foucault's term in \texit{Archaeology of Knowledge} to describe the specific forms of accumulation of a group of statements. }}
\newglossaryentry{referential}{name={referential}, description={For Michel Foucault, the referential of a statement is not the referent (the facts, things, realities or beings designated) but the place, condition, field of emergence or principle of differentiation for the entities named, described or designated in the statement. The referentials for machine learning include various hyperobjects such as genomes, social media, epidemics, markets and economies. Such referentials encompass many named entities. In this book, referential expands to include collective hyperobjects with their pre-individual reserves, and potentials for individuation.}}
\newglossaryentry{sigma}{name={\ensuremath{\sigma}}, description={An operator that sums together all the terms to the right of this symbol}}
\newglossaryentry{statement}{name={statement}, description={Michel Foucault's term for the result of an enunciative function that operationally relates a number of elements to a field of objects, establishing subject positions associated with them, and configuring a domain of coordination in which these elements can be invoked, used, and repeated. }}
\newglossaryentry{vectorised}{name={vectorised}, description={Operations on data that transform vectors of values.},see=[see also]{vector}}
\newglossaryentry{machine learner}{name={machine learner}, description={refers to both humans and machines involved in learning from data together.}}
\newglossaryentry{operational formation}{name={operational formation}, description={is a variation on Michel Foucault's discursive formation that highlights the collective human-machine regularities of power-knowledge. While operation and operational fields are intrinsic to Foucault's account of discursive practice, they are somewhat overshadowed by the figures of the document, the utterance, and the proposition.}}
