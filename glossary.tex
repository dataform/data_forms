\newglossaryentry{archaeology}{name={archaeology}, description={Michel Foucault defines archaeology as a description that explores the production of statements at the level of knowledge practices (\textit{savoir}). It emphasizes the irregularities and discontinuities in knowledge practices as well as the derivations of operations and functions}}
\newglossaryentry{beta}{name={\ensuremath{\hat{\beta}}}, description={is a commonly used symbol for the model parameters, weights or coefficients. Estimating optimum values of $\beta$ is a preoccupation in machine learning}}
\newglossaryentry{bias}{name={bias}, description={of a model refers to its inevitable approximation and misalignment to the actual processes that generated the data}}
\newglossaryentry{classifier}{name={classifier}, description={is a  machine learner that assigns instances to classes or categories such as \texttt{survive} or \texttt{die}, \texttt{cat} or \texttt{dog} }}
\newglossaryentry{cost function}{name={cost function}, description={is a  function that measures the difference between the output of the model (the prediction) and the known values (Whitehead,1960)}}
\newglossaryentry{cross-validate}{name={cross-validate}, description={is an operation that validates a model against a part of the data in order to gauge how well predictions generalize to fresh or hitherto unseen data. Many rounds of cross-validation may be used in training models when data is limited.}}
\newglossaryentry{data strain}{name={data strain}, description={borrows from A.N. Whitehead's notion of strain, which refers to implicit forces or tensions in bodies of data that relate to the feeling of geometrically straight or flat loci}}
\newglossaryentry{decision boundary}{name={decision boundary}, description={is a  boundary or surface drawn in  vector space by a machine learning classifier to differentiate or separate and hence classify cases}}
\newglossaryentry{deep learning}{name={deep learning}, description={a neural network comprising many layers commonly used for image recognition}}
\newglossaryentry{diagram}{name={diagram}, description={is a form of abstraction concerned with functioning and operations.  In Gilles Deleuze's reading of Michel Foucault, diagrams display relations of force,  and construct models of truth (Deleuze,1994)}}
\newglossaryentry{discourse}{name={discourse}, description={For Michel Foucault, a discourse groups statements generated by an enunciative function}}
\newglossaryentry{enunciative function}{name={enunciative function}, description={For Michel Foucault, the mapping of statements to themselves, to subject positions, to correlate domains and their material forms of reuse, replication and transcription together generate statements. In this book, the many predictions, inferences, plots, tabulations, numbers, scores, probabilities, classifications, software libraries and devices comprise the enunciative function of machine learning }}
\newglossaryentry{enunciative modality}{name={enunciative modality}, description={For Michel Foucault, the sites,  forms of observing, describing, teaching, perceiving associated with statements}}
\newglossaryentry{feature}{name={feature}, description={Also known in machine learning as variable, measurement, observation or attribute, a feature occupies one dimension in the vector space inhabited by data }}
\newglossaryentry{function}{name={function}, description={Mathematically, a function uniquely maps  one set of numbers onto another set of numbers.  each other. In machine learning, functions operate diversely, sometimes transforming data to generate feature or vector spaces, sometimes measuring cost or loss for particular models, and sometimes expressing forms such as curves and surfaces that transform data. Across these different usages and domains, the operation of mapping or relation between sets of values such as $X$ and $Y$ can be seen}}
\newglossaryentry{generative model}{name={generative model}, description={uses probability distributions to model the process that generated the data, thus allowing the model to generate or simulate samples from the data}}
\newglossaryentry{machine learner}{name={machine learner}, description={refers to humans and machines involved in learning from data together}, plural={machine learners}}
\newglossaryentry{Naive Bayes}{name={Na\"{\i}ve Bayes}, description={is a classifier that calculates the probability of an instance belonging to each class, and then assigns the instance to the most probable class}}
\newglossaryentry{operational formation}{name={operational formation}, description={is a variation on Michel Foucault's discursive formation that highlights the collective human-machine regularities of power-knowledge. While operation and operational fields are implicit to discursive practice, they are somewhat overshadowed by the figures of the document, the utterance, and the proposition in Foucault's account}}
\newglossaryentry{partial derivative}{name={partial derivative}, description={is an operator from differential calculus that expresses the rate of change of one variable with respect to another}, plural={partial derivatives}}
\newglossaryentry{partial observer}{name={partial observer}, description={in Gilles Deleuze and Félix Guattari's concept of what a mathematical function does in science (Deleuze,1994)}, plural={partial observers}}
\newglossaryentry{perceptron}{name={perceptron}, description={A machine learner developed in the 1950s by Frank Rosenblatt. It is modelled on a neurone that learns to classify the input data or what it 'perceives' by varying parameters or weights on the  sum of its inputs to produce values of either `1` or 0'}}
\newglossaryentry{probabilization}{name={probabilization}, description={The process where machine learners themselves are constituted as a population of devices, whose distribution and tendencies can be treated statistically}} 
\newglossaryentry{positivity}{name={positivity}, description={Michel Foucault's term in \textit{Archaeology of Knowledge} to describe the specific forms of accumulation of a group of statements in a discursive formation}}
\newglossaryentry{referential}{name={referential}, description={: for Michel Foucault, the referential of a statement is not the referent (the facts, things, realities or beings designated) but the place, condition, field of emergence or principle of differentiation for the entities named, described or designated in the statement. The referentials for machine learning include various hyperobjects such as genomes, social media, epidemics, markets and economies. Such referentials encompass many named entities }}
\newglossaryentry{regularization}{name={regularization}, description={ operates on the referentials of machine learning to target subtle, diffuse distributions of difference in order to classify, estimate and rank their effects}}
\newglossaryentry{statement}{name={statement}, description={Michel Foucault's term for the product of an enunciative function that operationally relates a number of elements to a field of objects, establishing subject positions associated with them, and configuring a domain of coordination in which these elements can be invoked, used, and repeated. Statements take many forms including utterances, graphs, equations and numbers (Foucault,82)}}
\newglossaryentry{sum}{name={\ensuremath{\sum}}, description={is an operator that sums together all the terms to the right of the symbol}}
\newglossaryentry{variance}{name={variance}, description={of a model refers to its dependence on the particular data it is trained on}}
\newglossaryentry{vector space}{name={vector space}, description={is a hyperspace of indefinite dimensions generated by the projective mapping of data variables or features into distinct coordinate  dimensions}}
\newglossaryentry{vectorize}{name={vectorize}, description={operations on data that transform vectors of values in aggregate},see=[see also]{vector}}
\newglossaryentry{vector}{name={vector}, description={ Three senses of the term are relevant: 1. A vector as an element of vector space; 2. A data structure in programming languages such as \texttt{R} -- a one dimensional array of elements; 3. A feeling in the sense used by A.N. Whitehead to describe the transfer from 'there' to 'here'}}
