\documentclass[../book.tex]{subfiles}

\chapter{$N = \forall\boldsymbol{X}$ Probabilisation and the Taming of Machines}
\label{ch:probability}

In the final pages of \emph{The Taming of Chance}, Ian Hacking describes
the work of the philosopher Charles Sanders Peirce in terms of a twin
affirmation of chance. First, Peirce, following the work of the
psychophysicist Gustav Fechner and before him the astronomer-sociologist
Adolphe Quetelet, re-enacts the normal curve.\footnote{The historian of
  statistics Stephen Stigler provides a lengthy account of Fechner's
  work in \autocite[239-259]{Stigler_1986}.} The `personal equation,'
the variation in measurements made by any observer, becomes `a reality
underneath the phenomena of consciousness' \autocite[205]{Hacking_1990}.
\index{Hacking, Ian!\textit{The Taming of Chance}} Peirce's belief in
absolute chance or a stochastic ontology, \index{ontology!stochastic} `a
universe of chance' as Hacking puts it, continued a series of
`realizations' of curves, in which astronomical, social, biological and
finally psychological variations were all understood as generated by
processes of chance. Second, and in order to show the underlying reality
of the normal curve, `Peirce deliberately used the properties of chance
devices to introduce a new level of control into his experimentation.
Control not by getting rid of chance fluctuations, but by adding some
more' \autocite[205]{Hacking_1990}. In the century or so since, what
happened to the thorough-going affirmation of statistical thought and
probabilistic practice epitomised by Peirce?
\index{Peirce, Charles Sanders!chance} Hacking stresses that he does not
understand Peirce as the precursor or the innovator of twentieth century
statistical thought (Hacking's \emph{Taming of Chance} ends at 1900),
but rather as `the first philosopher to conceptually internalize the way
chance had been tamed in the nineteenth century' (215). What would the
equivalent philosopher-machine learner internalize today? What would
such persons, working in science or media or government, hold firm in
relation to chance, probability and statistics?
\index{Hacking, Ian!on C.S. Peirce}

In the opening lines of the preface to the First Edition of
\emph{Elements of Statistical Learning}, Hastie, Tibshirani and Friedman
describe the altered situation of statistics:

\begin{quote}
The field of Statistics is constantly challenged by the problems that
science and industry brings to its door. In the early days, these
problems often came from agricultural and industrial experiments and
were relatively small in scope \autocite[xi]{Hastie_2009}
\index{\textit{Elements of Statistical Learning}!on statistics}
\end{quote}

(At the end of the preface, they also cite, we might note in passing,
Hacking's work: `The quiet statisticians have changed our world'
\autocite[xii]{Hastie_2009}.) One of the challenges science and industry
has brought to the door of statistics in recent years has not only been
more data but also machine learners. What difference do the `vast
amounts of data \ldots{} generated in many fields' (xi) make to the
field of statistics? Statistics has, I will suggest in this chapter,
gradually \emph{probabilised} machine learners, or injected a substratum
of chance that flows directly from their operation.
\index{probabilisation} To grasp this \gls{probabilisation}, we need to
determine what role randomness, change and the probabilistic
distribution of elements and events play in machine learning. These
questions of how worlds becomes thinkable through machine learning can
be addressed partly by contrasting the `taming of chance' achieved by
eighteenth and nineteenth century statistics and the taming of data --
and machines -- in statistical practices of machine learning today.
\index{machine learning!statistical practices}

\section{Data reduces uncertainty?}\label{data-reduces-uncertainty}

The broadest claim associated with machine learning hinges on the simple
expression shown:

\begin {equation}
\label {eq:n_all}
N = \forall\boldsymbol{X}
\end {equation}

In Equation \ref{eq:n_all}, \(N\) refers to the number of observations
(and hence the size of the dataset), the logical operator \(\forall\)
means `all' since this is the level of inclusion which many fields of
knowledge in science, government, media, commerce and industry envisage,
and \(\boldsymbol{X}\) refers to the data itself arrayed in vector
space. Note that this expression leaves some things out. \(Y\), the
response variable, for instance, may or may not be known or part of the
data \(\boldsymbol{X}\). \index{data!all of} While both the expansion of
data in the vector space and the machine learners that transform and
observe it have appeared in previous chapters, I focus here on changes
in probability practices associated with machine learning, and in
particular, \(N = \forall\boldsymbol{X}\), the claim that with all the
data, the production of knowledge fundamentally changes.
\index{'Big Data'|see {data!all of}}

The claim that with \(N=\forall\boldsymbol{X}\) the nature of knowledge
changes has been widely discussed.\footnote{Rob Kitchin provides a very
  useful overview of these claims in \autocite{Kitchin_2014}. While I
  will not analyse the claims about `big data' in specific cases in any
  great detail, the growing literature on this topic suggests that
  machine learning in its various operations -- epistopic construction
  of vector space, function finding as association of partial observers
  and a re-internalisation of probability -- generates considerable
  difficulties and challenges for knowledge, power and production.
  \index{Kitchin, Rob!on big data}} Viktor Mayer-Schönberger and Kenneth
Cukier's \emph{Big Data: A Revolution That Will Transform How We Live,
Work and Think} present this shift in many different settings in the
course of the vignettes and teeming comparisons that have become typical
of the data revolution genre. In a chapter entitled `More,' they sketch
the transition from data practices reliant on sampling to data practices
that deal with all the data:\index{data!sampling!limits of}

\begin{quote}
Using all the data makes it possible to spot connections and details
that are otherwise cloaked in the vastness of the information. For
instance, the detection of credit card fraud works by looking for
anomalies, and the best way to find them is to crunch all the data
rather than a sample \autocite[2013, 27]{Mayer-Schonberger_2013}
\end{quote}

In the several hundred pages that follow in \emph{Big Data}, the problem
of how to `crunch all the data' is not a major topic. Machine learning
remains almost completely invisible as a practice of transforming data
in the name of knowledge. \index{Mayer-Schőnberger, Viktor}
\index{Cukier, Kenneth} While they mention the role of social network
theory (30), `sophisticated computational analysis' (55), `predictive
analytics' (58) and `correlations' (7), and they observe that `the
revolution' is `about applying math to huge quantities of data in order
to infer probabilities' (12), any further consideration of a change in
data practices is largely confined to a business-oriented contrast
between having some of the data and having all the data (that is,
businesses often have all the data on their customers).

Without a sense of how statistical practices animate and configure key
features of `crunching the data' to make predictions, it becomes hard to
see how the `revolution' takes place. Just as nineteenth century
statistics transformed many measurements into population attributes (for
instance, mean as the ideal or abstract property of a population), the
shift between \(n\) and \(\forall\boldsymbol{X}\), between some and all,
\index{statistics!mean} a shift very much dependent on machine learning,
internalizes, I will suggest, population attributes into the operations
of machine learners.\index{machine learner!population of} This is a
statistical event akin to the advent of the Normal distribution (and
indeed, \(\mathnormal{N}\) is a standard symbol for the Normal
distribution in statistics textbooks) as a way of knowing and
controlling populations \autocite[108]{Hacking_1975}. To signal its
continuity with the invention of probability, I term it
`probabilisation,' a pleonasm that refers that facet of the operational
formation that renders knowledge in terms of probabilities.
\index{probabilisation}

\section{Machine learning as statistics inside
out}\label{machine-learning-as-statistics-inside-out}

The argument mimics Hacking's. In \emph{The Taming of Chance}, Hacking
argues that modern statistical thought transposed a way of calculating
errors in experimental measurements and astronomical observations into
the real and constitute attributes of populations understood as
processes of reproductive growth.
\index{statistics!history of!from error to real quantity} This
transposition or inversion relied on four intermediate steps passing
through the development of a probability calculus (particularly the work
of Jacob Bernoulli and the binomial or heads-tails probability
distribution in the 1690s \autocite[143]{Hacking_1975}), the
accumulation of large numbers of measurements (the most famous being the
chest measurements of soldiers in Scottish regiments, but these were
only one flurry amidst an avalanche of numbers in the 1830-1840s), the
emergence of the idea of multiple, minute independent causes producing
events (particularly as developed in medicine but also in studies of
crime), and the `law of errors' applying to measurements made by,
amongst others, astronomers
\autocite[111-112]{Hacking_1990}.\index{dataset!Scottish chest measurements}
\index{statistics!probability distributions!normal}
\index{statistics!measurements in} As Hacking observes, coins, suicides,
crime, chest measurements, and astronomical observations all pile up in
a statistical aggregate which remains, although somewhat altered,
indelible in contemporary statistical knowledges, particularly in its
frequent recourse to notions of population, probability and
distribution. In this entanglement, observers and the observed changed
places. The distribution of errors made by astronomers measuring the
position of stars or planets became a distribution or variation inherent
in a population. \index{population}

Machine learning reverse-engineers the invention of modern statistical
thinking. It takes back the `real quantities' -- probabilities -- that
modern statistics had attributed to the populations in the world and
distributes them to devices, to machine learners that people then
observe, monitor and indeed measure again in many ways. The direct
swapping between uncertainty in measurement and variation in real
attributes that statistics achieved now finds itself re-routed and
intensified as machine learners measure the errors, the bias and the
variance of devices. Although it relies heavily on probability
distributions, machine learning is a fat-tailed distribution of
probability.

The swapping or re-distribution is not a simple mirror-image reversal,
as if machine learners mistake devices for a population. Machine
learning constantly takes statistical thinking as a basic condition for
its operations and devices. When \emph{Elements of Statistical Learning}
states that (as we saw in the previous chapter) `our goal is to find a
useful approximation \(\hat(f)(x)\) to the function \(f(x)\) that
underlies the predictive relationship between input and output'
\autocite[28]{Hastie_2009}, they invoke the `real quantities' first
elaborated and articulated by proto-statisticians such as Quetelet
grappling with population and sample parameters. The major structuring
operational practices in machine learning as a field of
knowledge-practice show the marks of increasingly strong commitment to
the reality of the statistical, and to the ongoing probabilisation of
machine learners. \index{probabilisation!as statistical practice}

What is probabilisation in practice? Reading and working with machine
learning techniques usually means encountering and responding to
apparatus drawn from statistics, but the apparatus is not typically the
statistical tests of significance or variation. In contrast to a
statistics textbook such as the widely used \emph{Basic Practice of
Statistics} \autocite{Moore_2009} or a more advanced guide such as
\emph{All of Statistics} \autocite{Wasserman_2003}, where statistical
tests (t-test, chi-squared test, etc.), hypothesis testing, and analysis
of uncertainties (confidence intervals, etc.) order the exposition,
\index{statistics!textbok} machine learning textbooks rely on a
conceptual apparatus curiously stripped of statistical tests and
measurements. Statistical underpinnings may be fundamental, but this
does not mean that machine learners simply automate statistics.

\begin{table}
\centering
\begin{tabular}{|l|l|}
parametric &  non-parametric \\
bias &        variance \\
prediction &  inference \\
generative &  discriminative \\
\end{tabular}
\caption{Some structuring differences in machine learning}
\label{tab:ml_diffs}
\end{table}

Instead, a basic set of contrasts or indeed oppositions that owe much to
probabilistic thinking order, compose, associate and link the statements
of machine learners. \index{machine learning!structure differences} The
contrasts shown in Table \ref{tab:ml_diffs} all have a statistical facet
and anchoring to them. Some refer to errors that affect how a machine
learner refers to data (bias and variance; see discussion below); some
designate an underlying statistical intuition about how particular
machine learners treat data (does the model seek to generate the data or
classify -- discriminate -- it; e.g.~Naive Bayes or Latent Dirichlet
Allocation are \gls{generative} models whereas logistic regression or
support vector machines are \emph{discriminative});
\index{model!generative} \index{model!discriminative} parametric and
non-parametric describe the role of probability distributions in the
model; \index{model!parametric and non-parametric} and others indicate
different kinds of statistical knowledge practice (prediction seeks to
anticipate while inference seeks to interpret, etc.; also see discussion
below). \index{machine learning!structuring differences} These broad
structuring differences reach down deeply into the architecture, the
diagrams, the practices, statements and visual objects and computer code
associated with \(N=\forall\boldsymbol{X}\). Because they anchor basic
operations of machine learning in probability, formalisms derived from
statistics have in the last two decades increasingly populated the
field, furnishing and rearranging its diagrammatic references
\index{diagram!reference} to the worlds of industry, agriculture, earth
science, genomics, etc., but also, crucially, triggering ontological
mutations in machine learners themselves.
\index{machine learning!probabilisation of|seealso{probabilisation}}

\section{Distributed probabilities}\label{distributed-probabilities}

While these structuring differences deeply shape practice in machine
learning, the underlying operator that allows swapping between knowledge
and the world, between events and devices, is probability, and in
particular, functions that describe variations in populations,
probability distributions. \index{function!probability distributions}
Probability distributions both map population variations and, as we will
see, multiply the number of things that count as populations.
\index{population!as probability distribution}
\index{probabilisation!as distributed probability}

The normal distribution pervades nineteenth century statistical thinking
as it affects populations across law, medicine, agriculture, finance and
not least, sociology as a domain of knowledge. Normal distributions
appear in countless variations in scientific, government and
institutional settings as functions that map events, measurements,
observations and records to evidential probability quantities.\footnote{Statistical
  graphics have a rich history and semiology that I do not discuss here
  (see \autocite{Bertin_1983}).}

\begin {equation}
\label {eq:gaussian_distribution}
f(x;\mu, \sigma^2) = \frac{1}{\sigma\sqrt{2\pi}}e^{-\frac{1}{2}(\frac{x-\mu}{\sigma})^2}
\end {equation}

The function shown in equation (\ref{eq:gaussian_distribution})
expresses the probability of a given value of the variable \(x\) given a
population whose variations (with respect to \(x\)) can be expressed in
terms of two parameters, \(\mu\) and \(\sigma\), the mean and variance.
This is the so-called normal or Gaussian distribution.\footnote{Dozens
  of differently shaped probability distributions map continuous and
  discrete variations to real numbers. Other probability distributions
  --- normal (Gaussian), uniform, Cauchy exponential, gamma, beta,
  hypergeometric, binomial, Poisson, chi-squared, Dirichlet,
  Boltzmann-Gibbs distributions, etc. (see \autocite{NIST_2012} for a
  gallery of distributions) --- functionally express widely differing
  patterns. \index{function!probability distribution!variety of} The
  queuing times at airport check-ins do not, for instance, easily fit a
  normal distribution. Statisticians model queues using a Poisson
  distribution, in which, unfortunately for travellers, distributes the
  number of events in a given time interval quite broadly. Similarly, it
  might be better to think of the probability of rain today in
  north-west England in terms of a Poisson distribution that models
  clouds in the Atlantic queuing to rain on the northwest coast of
  England. (Rather than addressing the question of whether it will rain
  or not, a Poisson-based model might address the question of how many
  times it will rain today.)} Its mathematics were intensively worked
over during the late eighteenth and early nineteenth centuries in what
has been termed `one of the major success stories in the history of
science' \autocite[158]{Stigler_1986}. It has a power-laden biopolitical
history closely tied with knowledges and governing of populations in
terms of morality, mortality, health, and wealth (see
\autocite[113-124]{Hacking_1975}. \index{population} The key parameters
here include \(\mu\), the mean and \(\sigma\), the variance, a number
that describes the dispersion of values of the variable, \(x\) are.
These two parameters together describe the shape of the curve. Given
knowledge of \(\mu\) and \(\sigma\), the normal or Gaussian probability
distribution maps all outcomes to probabilities (or numbers in the range
\(0\) to \(1\)). Put statistically, functions such as the Gaussian
distribution probabilise events as random variables. Every variable
potentially becomes a function: `a random variable is a mapping that
assigns a real number to each outcome' \autocite[19]{Wasserman_2003}.
\index{random variable|seealso{function!probability distribution}}
\index{parameters!of a probability distribution}

The possibility of treating population variations as random variables,
that is, as probability distributions, was a significant historical
achievement, one that continues to develop and ramify.\footnote{The
  mapping that assigns numbers to outcomes (heads v. tails; cancer v.
  benign; spam v. not-spam) is a probability distribution. As I have
  argued in \autocite{Mackenzie_2015d}, random variables have become
  much more widespread in statistical practice due to changes in
  computational techniques. \index{random variable}} Random variables
distribute probability in the world. When conceptualised as real
quantities in the world rather than epiphenomenal by-products of
inaccuracies in our observations or measuring devices, probability
distributions weave directly into the productive operations of power.
\index{probabilisation!as distribution} Distribution in the sense of
locating, positioning, partitioning, sectioning, serialising or queuing
operations has received much more attention in critical thought
(particularly in the many uses of Foucault's concept of disciplinary
power \autocite{Foucault_1977}), but in almost every setting,
distribution in the sense of counting, apportioning and weighting of
different outcomes also operates. This constant interweaving of spatial,
architectural, logistical and functional processes has energised
statistical thought for several centuries.\footnote{`Distribution'
  pervades Foucault's account of power and knowledge from \emph{The
  Order of Things} \autocite{Foucault_1992} onwards.
  \index{Foucault, Michel!on distribution} Foucault treats distributions
  in several different ways: as spatial or logistical techniques, as
  mathematical orderings of large numbers of people or things, and as a
  methodological and theoretical framing device. In \emph{Discipline and
  Punish} \autocite{Foucault_1977}, the spatial sense prevails, but in
  later works, the population or demographic sense of distribution takes
  precedence \autocite{Foucault_1998}. Distribution certainly has
  theoretical primacy in his account of power: `relations of
  power-knowledge are not static forms of distribution, they are
  ``matrices of transformations''\,' \autocite[99]{Foucault_1998}.}
\index{distribution|see{function!probability distribution}} For
instance, given the normal distribution, it is possible, under certain
circumstances, to effectively subjectify someone on the spot. If an
educational psychologist indicates to someone that their intelligence
lies towards the left-hand side of the normal curve peak (and hence less
than the population mean), they quickly assign them to a potentially
institutionally and economically consequential trajectory. Since its
inception in the social physics of Adolphe Quetelet as a way of
referring to a property of populations, the normal curve has not only
described but modulated and re-shaped populations (in terms of health,
morality and wealth). \index{statistics!history!Quetelet, Adolphe}

If functions such as equation (\ref{eq:gaussian_distribution}) have
persisted for so long as elements of population governmentality or
biopolitics, \index{biopolitics!populations} what happen to them in
machine learning? The pages of a book such as \emph{Elements of
Statistical Learning} show many signs of an ongoing invocation of
probability distributions. We could simply observe their abundance.
Hastie and co-authors invoke probability distributions. They speak of
`Gaussian mixtures,' `bivariate Gaussian distributions,' standard
Gaussian,`'Gaussian kernels,' `Gaussian assumptions,' `Gaussian errors,'
`Gaussian noise,' `Gaussian radial basis function,' `Gaussian
variables,' `Gaussian densities,' `Gaussian process,' and so forth. (The
term `normal' appears in an even wider spectrum of similar guises.)
Events, things, properties, operations, functions, and attributes all
associate with probability distributions.
\index{function!probability distribution!Gaussian}

The multiple invocations of probability distributions attests to the
variety of events (occurrence of cancer, occurrence of the word `Viagra'
in an email, a click on a hyperlink, etc.) map to real numbers. Despite
the sometimes dense mathematical diagrammaticism, the term
\emph{distribution} emphasises a tangible and practically resonant way
of thinking about how events or possible outcomes shift about as the
parameters of a function vary.\footnote{Machine learners adjust these
  parameters in different ways. For instance, parametric and
  non-parametric models (see table \ref{tab:ml_diffs}) differ in that
  the former have a limited number of parameters and the latter an
  undefined number of parameters (for instance, Naive Bayes, \emph{k}
  nearest neighbours or support vector machine models). But both kinds
  assume that an underlying probability distribution -- a function,
  `unobservable' or not -- operates, even if it changes with new data. A
  probability distribution under these assumptions becomes the closest
  reality we have to whatever process generated all the variations in
  data gathered through experiments and observations. From a
  probabilistic perspective, the task of machine learning is to estimate
  the parameters (the mean \(\mu\) and variance \(\sigma\) in the case
  of Gaussian curve) that shape of the curve of the probability
  distribution.} Whatever inferences and predictions become possible,
probability distributions are a crucial control surface for machine
learning understood as a form of movement through data.
\index{machine learner!probability distribution as control surface} In
contrast to the endowment of living aggregates such as populations with
probability that we see in the biopolitical history of statistics (and
later in natural sciences such as physics and biology), statistical
machine learning increasingly constitutes devices as populations via
probability distributions. \index{probabilisation}

\section{Naive Bayes and the distribution of
probabilities}\label{naive-bayes-and-the-distribution-of-probabilities}

How could machine learners become a population? The mathematical
expression for one of the most popular of all machine learning
classifiers, the Naive Bayes classifier, stands out for its
probabilistic simplicity and seeming lack of `moving parts'.
\index{machine learner!Naive Bayes}

\begin {equation}
\label {eq:naive_bayes}
f_j(X) = \prod_{k=1}^{p}f_{jk}(X_k)
\end {equation}

\begin{quote}
\autocite[211]{Hastie_2009}\index{machine learner!Naive Bayes}
\end{quote}

Some machine learners are so simple that they can be implemented in a
few lines of code. Along with the perceptron, linear regression, and
\emph{k} nearest neighbours, the function shown in equation
(\ref{eq:naive_bayes}) is one of the simplest one to be found in most
machine textbooks yet easily adapts for high dimensional data, the kind
of data associated with contemporary network infrastructures, scientific
instruments, online communications and \(N = \forall\boldsymbol{X}\) in
general.\footnote{The other contender for simplest machine learner would
  be the also very popular \emph{k} nearest neighbours. As Hastie et.
  al. observe: `these classifiers are memory-based and require no model
  to be fit' \autocite[463]{Hastie_2009}. Like the Naive Bayes
  classifier, the equation for \emph{k} nearest neighbours is simple:

  \begin {equation}
  \label {eq:knn}
     \hat{Y}(x) =\frac{1}{k} \sum_{x_i \in \textit{N}_{k}(x)} y_i
  \end {equation}

  where \(\textit{N}_{k}(x)\) is the neighborhood of \(x\) defined by
  the \(k\) closest points \(x_{i}\) in the training sample
  \autocite[14]{Hastie_2009}.

  In equation \ref{eq:knn}, a parameter appears: \(k\), the number of
  neighbours. This contrasts greatly with the linear models discussed in
  chapters \ref{ch:vector} and \ref{ch:function} where the number of
  parameters \(p\) usually equals the number of variables in the dataset
  or dimensions in the vector space.
  \index{machine learner!\textit{k}-nearest neighbours}} Even though the
Naive Bayes classifier is one of the most popular machine learning
algorithms, it is more than 50 years old \autocite{Hand_2001}.

The key diagrammatic elements of the classifier in the equation are
\(\prod\), an operator that multiplies all the values of the matrix of
\(X\) values (from \(1\) to \(p\)) to generate a product. What product
does the Naive Bayes classifier produce? The expression \(f_j(X)\)
refers to a probability density; that is, it describes the probability
that a particular thing (a document, an image, an email message, a set
of URLs, etc.) belongs to the class of things \(j\).
\index{probabilisation!probability density} In constructing an estimate
of the probability that a given message, image or event is an instance
of class \(j\), \(p\) different features are taken into account. ( The
subscript \(k\) indexes the \(p\) dimensions of the vector space.) The
subscripts \(k=1\) on the \(\prod\) operator, and \(k\) on the data
\(X_k\) indicate that the Naive Bayes classifier makes use of a series
of features or variables in calculating the overall probability that a
given thing or observation belongs to a specific class. Put in the
language of probability calculus, the classifier produces a probability
density \(f_j(X)\) by calculating the \emph{joint probability} of all
the \emph{conditional} probabilities of the features or predictor
variables in \(X\) for the class \(j\). As \emph{Elements of Statistical
Learning} rather tersely puts it, `each of the class densities are
products of the marginal densities' \autocite[108]{Hastie_2009}.

The Naive Bayes classifier directly invokes probability (including its
name, with its reference to the Bayes Theorem, an important late
eighteenth century concept), yet there is little obvious connection to
statistics in its modern form of tests of significance.
\index{statistics!Bayes Theorem} As Drew Conway and John Myles-White
write in \emph{Machine Learning for Hackers},

\begin{quote}
At its core, {[}Naive Bayes{]} \ldots{} is a 20th century application of
the 18th century concept of \emph{conditional probability}. A
conditional probability is the likelihood of observing some thing given
some other thing we already know about \autocite[77]{Conway_2012}
\index{Conway, Drew!on Naive Bayes}
\index{Myles-White, John!on Naive Bayes}
\end{quote}

They point to the application of `conditional probability,' a
probability conditioned on the probability of something else.
Conditional probability lies at the heart of many of the data
transformation associated with prediction or pattern recognition since
it links a class to the occurrence of combinations of variables or
features. Naive Bayes links variables by simply multiplying
probabilities.\footnote{In \autocite{Mackenzie_2014c}, I have suggested
  that the intensification of multiplication associated with
  probabilistic calculation may constitute an important mutation in the
  ontological and practical texture of numbers. The epidemiological
  modelling of H1N1 influenza in London 2009 involved multiplying a
  great variety of probability distributions in order to calculate the
  conditional probability of influenza over time.}
\index{probability!conditional} As any of the many accounts of the
technique will explain, the name comes from Bayes Theorem, one of the
most basic yet widely used results in probability theory (again dating
from the eighteenth century), yet Naive Bayes does not even fully
embrace Bayes Theorem as the principle of its operation. The classifier
has a simple architecture based on the concepts of conditional
probability and joint probability; it calculates a probability density
function \(f_j(X)\) or probability distribution for each possible class
of things as a combination of the probabilities of all the many features
or attributes of populations that come together in data. It makes a
drastically naïve assumption that features or variables are
statistically independent of each other, where `independent' means that
they do not affect each other, or that they have no relation to each
other. We will see below that dramatic simplifications such as
independence do not necessarily weaken the referential grasp of machine
learners on the world, but in certain ways allow them to reconfigure the
operations of machine learners as a population of learners.

\section{\texorpdfstring{Spam: when \(\forall{N}\) is too
much?}{Spam: when \textbackslash{}forall\{N\} is too much?}}\label{spam-when-foralln-is-too-much}

\(\forall{N}\) can be a bother.\index{data!all of!as a problem} In
\emph{Doing Data Science}, Rachel Schutt and Cathy O'Neill furnish a
\texttt{bash} script (that is, command line instructions) to download
the well-known\texttt{Enron} email dataset
\index{dataset!\texttt{Enron}} and build a Naive Bayes classifier that
labels email as spam or not. In many ways, this is canonical machine
learner pedagogy. For Naive Bayes, email spam detection has become the
standard example (Andrew Ng uses it in CSS229, Lecture 5
\autocite{Ng_2008}).\index{\textit{Doing Data Science}}
\index{Ng, Andrew!on spam email} In this setting, machine learners
operate as filters coping with too much communication.
\index{communication!too much}

A typical spam email in the \texttt{Enron} dataset, a dataset that
derives from the U.S Federal Energy Regulatory Commission's
investigation into Enron Corporation \autocite{Klimt_2004}, looks like
this:

\begin{quote}
Subject: it's cheating, but it works ! can you guess how old she is ?
the woman in this photograph looks like a happy teenager about to go to
her high school prom, doesn't she ? she' s an international,
professional model whose photographs have appeared in hundreds of ads
and articles whenever a client needs a photo of an attractive, teenage
girl.but guess what ? this model is not a teenager ! no, she is old
enough to have a 7-year-old daughter.. she also says, " if it weren't
for this amazing new cosmetic cream called `deception,' i would lose
hundreds of modeling assignments\ldots{}because\ldots{}there is no way i
could pass myself off as a teenager." service dept 9420 reseda blvd \#
133 northridge, ca 91324
\end{quote}

The text of a typical non-spam email like this:

\begin{quote}
Subject: industrials suggestions\ldots{}\ldots{}
----------------------forwarded by kenneth seaman / hou / ect on 01 / 04
/ 2000 12 : 47 pm-------------------------- - pat clynes @ enron 01 / 04
/ 2000 12 : 46 pm to : kenneth seaman / hou / ect @ ect, robert e lloyd
/ hou / ect @ ect cc : subject : industrials ken and robert, the
industrials should be completely transitioned to robert as of january 1,
2000.please let me know if this is not complete and what else is left to
transition . thanks, pat
\end{quote}

Such communications, with their mixture of solicitation and imperative
are familiar to anyone who uses email. How does Naive Bayes probablise
their differences? \index{differences!probabilisation of} How do they
become \(X\) or even \(f_j(X)\) in the Naive Bayes classifier? The code
that \emph{Doing Data Science} supplies is instructive:

\begin{lstlisting}[language=bash, caption={A Naive Bayes classifiers for \texttt{enron} email}, label={lst:nb_spam}]
    #!/bin/bash
    # description: trains a simple one-word naive bayes spam
    # filter using enron email data
    # usage: ./enron_naive_bayes.sh <word>
    # author: jake hofman (gmail: jhofman)

    ### PART 1
    Nspam=`ls -l spam/*.txt | wc -l`
    Nham=`ls -l ham/*.txt | wc -l`
    Ntot=$Nspam+$Nham
    echo $Nspam spam examples
    echo $Nham ham examples

    Nword_spam=`grep -il $word spam/*.txt | wc -l`
    Nword_ham=`grep -il $word ham/*.txt | wc -l`
    echo $Nword_spam "spam examples containing $word"
    echo $Nword_ham "ham examples containing $word"

    ### PART 2
    Pspam=`echo "scale=4; $Nspam / ($Nspam+$Nham)" | bc`
    Pham=`echo "scale=4; 1-$Pspam" | bc`
    echo
    echo "estimated P(spam) =" $Pspam
    echo "estimated P(ham) =" $Pham
    Pword_spam=`echo "scale=4; $Nword_spam / $Nspam" | bc`
    Pword_ham=`echo "scale=4; $Nword_ham / $Nham" | bc`
    echo "estimated P($word|spam) =" $Pword_spam
    echo "estimated P($word|ham) =" $Pword_ham

    ### PART 3
    Pspam_word=`echo "scale=4; $Pword_spam*$Pspam" | bc`
    Pham_word=`echo "scale=4; $Pword_ham*$Pham" | bc`
    Pword=`echo "scale=4; $Pspam_word+$Pham_word" | bc`
    Pspam_word=`echo "scale=4; $Pspam_word / $Pword" | bc`
    echo
    echo "P(spam|$word) =" $Pspam_word
    cd ..
\end{lstlisting}

\autocite[105-106]{Schutt_2013}

The script draws out something of how the joint probability function in
equation (\ref{eq:naive_bayes}) probabilises a single word.\footnote{The
  input to the script is a single word such as `finance' or `deal'. The
  model is so simple that it only classifies a single word as spam. The
  \texttt{bash} script carries out four different transformations of the
  data in building the model. It uses only command line tools such as
  \texttt{wc} (word count), \texttt{bc} (basic calculator),
  \texttt{grep} (text search using pattern matching) and \texttt{echo}
  (display a line of text). These tools or utilities are readily
  available in almost any UNIX-based operating system (e.g.~Linux,
  MacOS, etc.). The point of using only these utilities is to illustrate
  the simplicity of the algorithmic implementation of the model. The
  first part of the code downloads the sample dataset of Enron emails
  (and I will discuss spam emails and their role in machine learning
  below). Note that this dataset has already been divided into two
  classes - `spam' and `ham' -- and emails of each class have been
  placed in separate directories or folders as individual text files.
  \index{code!command line} \index{Unix}} \index{diagram!world} Not all
machine learning models are so simple that they can be conveyed in 30
lines of code (including downloading the data and comments),
\index{code!brevity} but the script signals that nothing that occurring
in probabilisation \index{probabilisation} is intrinsically mysterious,
elusive or indeed particularly abstract.\footnote{After fetching the
  dataset from a website, the code excerpted in \ref{lst:nb_spam} counts
  the number of emails in each category \texttt{spam} or \texttt{ham},
  and then counts the number of times that the chosen word (e.g.
  `finance' or `deal') occurs in both the spam and non-spam or ham
  categories. In Part 2, using these counts the script estimates
  probabilities of any email being spam or ham, and then given that
  email is spam or ham, that the particular word occurs. (To estimate a
  probability means, in this case, to divide the word count for the
  chosen word by the count of the number of spam emails, and ditto for
  the ham emails.) In Part 3, the final transformation of the data,
  these probabilities are used to calculate the probability of any one
  email being spam given the presence of that word. Again, the
  mathematical operations here are no more complicated than adding,
  multiplying and dividing. The probability that the chosen word is a
  spam word is, for instance, the probability of occurrence of the word
  in a spam email multiplied by the overall probability that an email is
  spam. Finally, given that the probability of the chosen word occurring
  in the email dataset is the probability of it occurring in spam plus
  the probability of it occurring in ham, the overall probability that
  an email in the Enron data is spam given the presence of that word can
  be calculated. (It is the probability that the chosen word is a spam
  word divided by the probability of that word in general.)} On the
contrary, the power of classifiers operates through the accumulated
counting, adding, multiplying (that is, repeated adding) and dividing
(that is, multiplying by parts or fractions) constrained by the joint
probability distribution. Probability re-distributes things such as
emails or documents as, in this case, events in a population of words.
The Naive Bayes classifies endows every word in the \texttt{Enron}
dataset with a probability density function. The classification of each
email becomes a matter of estimating a conditional probability based on
the joint probability distribution that quantifies the chance of all the
words in that email appearing together. Probabilities are always between
\texttt{0} and \texttt{1}, and classification entails selected a cutoff
or dividing line. For instance, greater than \texttt{0.5} might result
in a classification as \texttt{spam}. In the \texttt{enron} dataset,
`finance' has a 0.69 chance of being spam, while `sexy' has a chance of
1. Ironically, like the Naive Bayes classifier's own reliance on
seventeenth and eighteenth century probability calculus, the frequent
application of this machine learner to document classification and
retrieval echoes the seventeenth century thinking that first conceived
of the very notion of `probability' in relation to the evidential weight
of documents \autocite[85]{Hacking_1975}. \index{probability!history of}

\section{The improbable success of the Naive Bayes
classifier}\label{the-improbable-success-of-the-naive-bayes-classifier}

There is something quite artificial at work in the construction of these
populations and their associated probability distributions.
\index{probabilisation!construction of populations} They are
intentionally artificial and limited. They do not correspond or refer
directly to what we know, for instance, of how language works, but
instead to a rather different set of concerns. Like most machine
learning techniques encountering complex realities, classifiers such as
Naive Bayes ignore many obvious structural or semiotic features of
emails as documents (for instance, word order, or co-occurrences of
words). Yet this very artificiality or limitation in their reference to
the world allows machine learners to appear in many different guises.
Despite their simple architecture, Naive Bayes classifiers have been
surprisingly successful. Many machine learners transform vectorised data
into probability distributions populated by fields of random variables
in process of change. They render all things as populations.

\begin{table}[ht]
\centering
\begingroup\tiny
\begin{tabular}{p{0.05\textwidth}p{0.10\textwidth}p{0.85\textwidth}}
  \hline
 & Year & Title \\ 
  \hline
812 & 2002 & On Discriminative vs. Generative classifiers: A comparison of logistic regression and naive Bayes \\ 
  140 & 2004 & Molecular similarity searching using atom environments, information-based feature selection, and a naive Bayesian classifier \\ 
  808 & 2004 & Some theory for Fisher's linear discriminant function, 'naive Bayes', and some alternatives when there are many more variables than observations \\ 
  299 & 2004 & Enrichment of extremely noisy high-throughput screening data using a naive Bayes classifier \\ 
  813 & 2004 & Augmenting naive Bayes classifiers with statistical language models \\ 
  815 & 2004 & Combination of a naive Bayes classifier with consensus scoring improves enrichment of high-throughput docking results \\ 
  298 & 2005 & Not so naive Bayes: Aggregating one-dependence estimators \\ 
  809 & 2006 & Prediction of protein homo-oligomer types by pseudo amino acid composition: Approached with an improved feature extraction and Naive Bayes Feature Fusion \\ 
  126 & 2006 & Combining multi-species genomic data for microRNA identification using a Naive Bayes classifier \\ 
  293 & 2006 & Enrichment of high-throughput screening data with increasing levels of noise using support vector machines, recursive partitioning, and Laplacian-modified naive Bayesian classifiers \\ 
  120 & 2008 & Ligand-Target Prediction Using Winnow and Naive Bayesian Algorithms and the Implications of Overall Performance Statistics \\ 
  225 & 2009 & Feature selection for text classification with Naive Bayes \\ 
   \hline
\end{tabular}
\endgroup
\caption{Most cited Naive Bayes publications 1945-2015} 
\label{tab:nb_most_cited}
\end{table}

The altered relation between modern statistical and machine learning
practice starts to appear in Naive Bayes from the early 1990's as
statisticians begins to generalize and re-diagram Naive Bayes by
examining its statistical properties more carefully. Table
\ref{tab:nb_most_cited} shows 30 of the most cited Naive Bayes-related
scientific publications.\footnote{Citation counts, even from the more
  reliable Reuters-Thomson Web of Science database, are difficult to
  evaluate when moving between disciplines. Some fields, such as
  computer science and biology, publish huge numbers of papers compared
  to smaller disciplines such as astronomy or plant ecology.} The list
of titles sketches a double movement. On the one hand, we see the
typical diagonal forms of accumulation or positivity \index{positivity}
of a machine learner across disciplines -- computer science, statistics,
molecular biology (especially of cancer), software engineering, internet
portal construction, sentiment classification, and image `keypoint'
recognition. On the other hand, highly cited papers such as
\autocite{Friedman_1997} and \autocite{Hand_2001} point to an
intensified statistical treatment of machine learners during these
years, an intensified probabilisation of machine learners that strongly
affects their ongoing development (leading, for instance, to the much
more document-oriented, heavily probabilistic topic models appearing in
the following decade \autocite{Blei_2003}).
\index{machine learner!topic model}

In \emph{Elements of Statistical Learning}, Hastie, Tibshirani and
Friedman characterise the Naive Bayes classifier in terms of its
capacity to deal with high dimensional data:

\begin{quote}
It is especially appropriate when the dimension \(p\) of the feature
space is high, making density estimation unattractive. The naive Bayes
model assumes that given a class \(G = j\), the features \(X_k\) are
independent \autocite[211]{Hastie_2009}.
\index{machine learner!Naive Bayes!success of}
\end{quote}

Similar formulations can be found in most of the machine learning books
and instructional materials. This appropriateness relates directly to
\(\forall{\boldsymbol{X}}\), and the expansion of the vector space. As
we saw above in equation (\ref{eq:naive_bayes}), \(p\) stands for the
number of different dimensions or variables in the data set. In the spam
classifier, the number of dimensions balloon hundreds of thousands
because every unique word adds a new dimension to the vector space.
Compared to the complications of logistic regression, neural networks or
support vector machines, \ref{eq:naive_bayes} seems incredibly simple.
How is it that a simple multiplication of probabilities and the
assumption that `features \ldots{} are independent' can, as Hastie and
co-authors write: `often outperform far more sophisticated alternatives'
\autocite[211]{Hastie_2009}?

The answer to this conundrum of success does not lie in the increasing
availability of data to train machine learners on. I want to explore two
other contrasts as ways of viewing the probabilising processes at work
in Naive Bayes. The first way to view this success is in terms of
\emph{ancestral communities} of probabilisation. The second concerns the
statistical decomposition of machine learners in terms of their sources
of error. \index{probabilisation!ancestral communities of}
\index{machine learner!statistical decomposition of}
\index{probabilisation!errors in}

\section{Ancestral probabilities in documents: inference and
prediction}\label{ancestral-probabilities-in-documents-inference-and-prediction}

Why is the Naive Bayes classifier is almost always demonstrated on the
problem of filtering spam email
\autocites{Conway_2012}[93-113]{Schutt_2013}[53]{Kirk_2014}[92-93]{Lantz_2013}{Flach_2012}{Ng_2008b},
and in particular dealing with the abundance of spam emails mentioning a
drug for erectile dysfunction sold under the tradename `Viagra' (a drug
that was itself the byproduct of the clinical trial for hypertension and
heart disease)? \index{machine learner!Naive Bayes!spam} What are we to
make of this regularity in production of statements? Admittedly spam,
and spam trying to sell Viagra in particular, has been a very familiar
part of most email since 1997 when Viagra was approved for sale, and of
all the documents that machine learners mundanely encounter in quantity
in those years, email might be the most numerous as well as one of the
mundanely shared. Naive Bayes classifiers and variations of them also
became practical devices in managing email traffic for most people,
whether they know it or not, during the mid-1990s (see for instance,
\href{http://spamassassin.apache.org/}{SpamAssassin}.(The other would be
scientific publications. Many more recent machine learners train as
classifiers on scientific publications \autocite{Blei_2007}
\index{science!publications!classification of})

From an archaeological standpoint, the reiteration of email spam
filtering using Naive Bayes is the effect of another process, a process
akin to the attribution of probability distributions to populations in
the nineteenth century. Like many machine learners, Naive Bayes has one
important lineage derived from the problem of classifying and retrieving
documents amidst archives. The operational practice of document
classification is specified in the element of the archive.
\index{data!archives of} Genealogical affiliation with a particular
problem such as document classification (or image recognition) generates
many re-iterations and versions of machine learners over time. As Lucy
Suchman and Randall Trigg wrote in their study of work on artificial
intelligence,

\begin{quote}
rather than beginning with documented instances of situated inference
\ldots{} researchers begin with \ldots{} postulates and problems handed
down by the ancestral communities of computer science, systems
engineering, philosophical logic, and the like
\autocite[174]{Suchman_1992}.
\index{artificial intelligence!ancestral communities in}
\end{quote}

While Bayes Theorem dates from the 18th century, the highly successive
use of Naive Bayes classifiers in email spam filtering in recent decades
effectively draws on an ancestral community of document classification
and information retrieval methods reaching back to the mid-20th
century.\footnote{The other lineage descends from medical diagnosis. For
  instance, starting in 1960, Homer Warner, Alan Toronto and George
  Veasy, working at the University of Utah and Latter-day Saints
  Hospital in Salt Lake City, began to develop a probabilistic computer
  model for diagnosis of heart disease
  \autocites{Warner_1961}{Warner_1964}. Their model used exactly the
  same `equation of conditional probability' we see in equation
  \ref{eq:naive_bayes} but now used to `express the logical process used
  by a clinician in making a diagnosis based on clinical data'
  \autocite[177]{Warner_1961}. Despite the mention of logic in this
  description, the diagnostic model was thoroughly probabilistic in the
  sense that the model itself has no representation of logic included in
  its workings. Rather it calculates the probability of a given type of
  heart disease given `statistical data on the incidence of symptoms'
  \autocite[558]{Warner_1964}. Somewhat ironically, as they point out,
  physicians involved in preparing and submitting data to the diagnostic
  program improved the accuracy in their own diagnoses. In 1964, N.J
  Bailey was taking the same approach to medical diagnosis
  \autocite{Bailey_1965}. \index{medical diagnosis}
  \index{machine learner!Naive Bayes!history of} Heart disease to a
  central topic in machine learning (see chapter \ref{ch:function} for
  discussion of the \texttt{South\ African\ Heart\ Disease} dataset
  \index{dataset!South African Heart Disease}).}
\index{Suchman, Lucy!on ancestral communities}

Early attempts to use what is now called Naive Bayes in the early 1960s
re-iterated engagements with the evidential weight of documents that
accompanied the emergence of probabilistic thinking as a quantification
of belief in the seventeenth century \autocite[35-49]{Hacking_1975}.
\index{probability!emergence of} Working at the RAND Corporation in the
early 1960s, M.E. Maron described how `automatic indexing' of documents
-- Maron used papers published in computer engineering journals -- could
become `probabilistic automatic indexing.' The necessary statistical
assumption was:

\begin{quote}
The fundamental thesis says, in effect, that statistics on kind,
frequency, location, order, etc., of selected words are adequate to make
reasonably good predictions about the subject matter of documents
containing those words \autocite[406]{Maron_1961} \index{Maron, M.E.}
\end{quote}

This thesis has remained somewhat fundamental in text classification and
information retrieval applications, as well as many other machine
learning approaches since. Maron's work focused on a collection of
several hundred abstracts of papers published in the March and June 1959
issues of the \emph{IRE Transactions on Electronic Computers}. As in
contemporary supervised learning, these abstracts were divided into two
groups, a training and a test set (`group 1' and `group 2' in Maron's
terminology \autocite[407]{Maron_1961}), and the training set was
classified according to 32 different categories that had already been in
use by the Professional Group on Electronic Computers, the publishers of
the \emph{IRE Transactions}. Given these classifications, word counts
for all distinct words in the abstracts were made, the most common terms
(`the', `is', `of', `machine', `data', `computer') and the most uncommon
words removed, and the remaining set of around 1000 words were actually
used for classification.

This treatment of the abstracts as documents, then as lists of words,
and then as frequencies of terms, and finally as a filtered list of most
information rich terms continues in much document and text
classification work today. \index{dataset!engineering paper abstracts} A
typical contemporary information retrieval textbook such as
\autocite{Manning_2008} devotes a chapter to the topic, including the
canonical discussion of how simplifying assumptions about language and
meaning do not vitiate the Naive Bayes classifier. Whenever machine
learners announce the unlikely efficacy of classifiers, we might attend
to the ways in which previous `ancestral probabilisations' and archival
constitution of the domain in question prepare the ground for that
success. \index{machine learner!Naive Bayes!history of}
\index{probabilisation!ancestral communities of}

\section{Statistical decompositions: bias, variance and observed
errors}\label{statistical-decompositions-bias-variance-and-observed-errors}

Even with an eye on the ancestral communities that constantly accompany
and heavily shape the indexical diagram of machine learning in the
world, we still need a way of accounting for the artificiality of Naive
Bayes. \index{diagram!indexical} The classifiers generates highly
arbitrary probabilities of document class membership, yet these
arbitrary probabilities still allow effective classification. Machine
learners view the persistence of manifest artifice (in the case of Naive
Bayes, a model that eschews any modelling of relations between things in
the word such as words) in terms of another of the structuring
differences of machine learning: the so-called
\emph{bias-variance\_decomposition}\autocite[24]{Hastie_2009}.
\index{error!bias-variance|(}

The terms `bias' and `variance' stem from the long history of
statistical interest in errors (as Hacking's account of the
transposition of measurement errors into population norms illustrates).
\index{statistics!errors} The \gls{bias} and \gls{variance} of
`estimators' -- the estimates of the parameters of the models usually
written as \(\hat{\beta}\) or \(\hat{\theta}\)-- feature heavily in
machine learning discussions of prediction errors. The terms point to
tensions that all machine learners experience. On the one hand,
\emph{variance} refers to the inevitable reliance of a machine learner
on the data it `learns.' To put it more formally, `variance refers to
the amount by which \(\hat{f}\) would change if we estimated it using a
different training data set' \autocite[34]{James_2013}. On the other
hand, \emph{bias} `refers to the error that is introduced by
approximating a real-life problem, which may be extremely complicated,
by a much simpler model' \autocite[35]{James_2013}.

These two sources of error, one which results from sampling and the
other arising from the structure of the model or approximating function,
can be reduced or at least subject to trade-off in what \emph{Elements
of Statistical Learning} terms `the bias-variance decomposition'
\autocite[223]{Hastie_2009}.\footnote{Another source of error, the
  `irreducible error' \autocite[37]{Hastie_2009} is noise that no model
  can eliminate.} From the standpoint of the bias-variance
decomposition, every machine learner makes a trade-off between the
errors deriving from differences between samples, and errors due to the
difference between the approximating function and the actual process
that generated the data. Note that both sources of error in the
bias-variance decomposition derive from transformations of the data.
Variance affects how the model encounters the world (as a set of small
samples or as, at the other end, a massive \(N=\forall{\boldsymbol{X}}\)
dataset). Bias relating to how the model `apprehends' the data (as a set
of almost coin-toss like independent events, as a geometrical problem of
finding a line or curve that runs through a cloud of points, etc.).

Even with all the data, machine learning cannot fully circumvent the
tensions between the different errors at work in the bias-variance
decomposition. \index{data!all of!insufficient} Yet, sources of error do
not always prove harmful. \index{error!value of} The success of Naive
Bayes (and \emph{k} nearest neighbours classifier) runs counter to the
long standing trend in statistics to construct increasingly
sophisticated models of the domains they encounter. Writing in 1997,
Jerome Friedman describes how very simple classifiers perform
surprisingly well:

\begin{quote}
Certain types of (very high) bias can be canceled by low variance to
produce accurate classification \autocite[55]{Friedman_1997}
\index{Friedman, Jerome!on bias-variance decomposition}
\end{quote}

A rather elaborate set of concepts and techniques address the
bias-variance decomposition in the context of data availability. These
techniques focus on managing the \emph{test} or \emph{generalization}
error, the difference between the actual and predicted values produced
by the machine learner when it encounters a fresh, hitherto unseen data
sample. Machine learners in such settings still encounter the
bias-variance trade-off as they select some data for training and some
data for testing. This trade-off has to deal with the fact that training
errors -- the observed difference between what the model predicts and
what the training data actually shows -- are not a good guide to test or
generalization error. \index{error!training}
\index{error!test|see {error!generalization}} The process of fitting a
model or finding a function (see previous chapter) will tend to reduce
the training error by fitting the function more and more closely to the
shape of the training data, but when it encounters fresh data that
function might no longer fit well. In other words, a more sophisticated
function may well reduce the bias but increase the variance. `Richer
collections of models' \autocite[224]{Hastie_2009} reduce bias, but tend
to increase variance. Conversely, models that cope well with fresh data
(and Naive Bayes is a good example of such a machine learner), display
low variance but high bias.

The trade-offs between bias and variance shift markedly between
different types of models, and generates many different conceptual
analyses of error in machine learning literature (`optimism of the
training error rate' (228), `estimates of in-sample prediction error'
(230), `Bayesian information criterion' (233), `Vapnik-Chervonenkis
dimension' (237), `minimum description length' (235)) and technical
methods of estimating prediction error (`cross-validation' (241),
`bootstrap methods' (249), `expectation-maximization algorithm' (272),
`bagging' (282), or `Markov Chain Monte Carlo (MCMC)' (279)), many of
which date from the 1970s (e.g.~cross-validation \autocite{Stone_1974},
bootstrap \autocite{Efron_1979}, expectation-maximization
\autocite{Dempster_1977}). \index{error!analysis of in machine learning}
\index{error!techniques of estimating}

A daunting field of concepts, themes, techniques and methods all
gravitate to the threshold of probabilisation.
\index{probabilisation!threshold of} They invoke in some cases
sophisticated mathematical or statistical constructs. They also very
often rely on computational iteration or infrastructural scale to
optimise parameters in models whose underlying intuitions remain quite
straightforward (as in a linear regression or Naive Bayes). In some
cases, the implementation of a model may be very simple, but analysis of
how the machine learner manages to curtail a source of error such as
bias or variance entails much more sophisticated statistical
understanding. Many analyses of how a model becomes a `useful
approximation' reconfigure the models themselves as members of a
population whose variations and uncertainties, whose tendencies and
predispositions must be sampled, tested and
monitored.\index{population!machine learners} The bias-variance
decomposition points to an irreducible friction in the way that machine
learning structures differences in the world.
\index{error!bias-variance|)}

\section{Does machine learning construct a new statistical
reality?}\label{does-machine-learning-construct-a-new-statistical-reality}

\index{differences!errors in} Following a broadly Foucaultean line of
argument, Hacking proposes that statistical thinking and practice in the
nineteenth and early twentieth century ontologically re-configured
things in terms of probability distributions (and the Gaussian
distribution in particular).\index{Hacking, Ian!statistics, history of}
What happens in worlds where the statistical treatment of error -- the
bias-variance decomposition is a shorthand term for this -- distributes
probability throughout an operational formation?
\index{operational formation!statistical composition of} I have
suggested that an ancestral probabilisation of domains and the
statistical decomposition of error come together in statistical machine
learning. The bias-variance decomposition includes both tightly bound
points and certainly relatively free or unbound points, as we saw in the
case of the Naive Bayes classifier in its encounter with data. It
generates highly erroneous probability estimates but performs well as a
classifier.

Viewed diagrammatically, unbound points matter greatly to the relations
of force at work in a knowledge-power conjunction. Probabilisation gives
machine learning a relation to its own plurality, to the tendencies of
its models to proliferate and vary.
\index{probabilisation!as relation to machine learning} Every attempt to
construct a machine learner in a given setting draws on both the
re-iteration of ancestral probabilities (that is, prior structuring of
settings in conformity with some probability distribution) or on the
many interactive adjustments, re-distributions and re-samplings of the
data \emph{and} transformations of the models associated with the
bias-variance decomposition.

Mayer-Schönberger and Cukier argue that having much data or all data
(\(N=\forall{\boldsymbol{X}}\)) re-bases knowledge. \index{data!all of}
Versions of this claim can be found running through various scientific
and business settings throughout the 20th century.\footnote{Later
  chapters of this book will track several instances of having all the
  data in the sciences, in government and in business in order to show
  what having all the data entails in different settings.
  \index{data!all of}} In certain settings, \(N=all\) has been around
for quite a while (as for instance, in many document classification
settings where the whole archive or corpus of documents have been
electronically curated for decades). Mayer-Schönberger and Cukier
rightly emphasize that the huge quantities of data sluicing through some
contemporary infrastructures support wider inferences (11). Their
discounting of statistical sampling as a concept `developed to solve a
particular problem at a particular moment in time under specific
technological constraints' \autocite[31]{Mayer-Schonberger_2013} does
not, however, accommodate the operational practices of sampling that
pervade machine learning, particularly in the forms of probabilisation.

Whether or not someone uses Naive Bayes, a topic model, neural networks
or logistic regression, does not greatly alter the processes of
probabilisation. Random variables, probability distributions, errors and
model selection practices crowd in around and re-configure machine
learners as members of a population generating statements.
\index{population!of machine learners} In many ways, the
Mayer-Schönberger and Cukier account bobs in the wake of the
enterprise-wide accumulations of data. They pay so much attention to the
capital potentials of data accumulation that they cannot easily attend
to the question of how machine learners probabilise that data. Sampling,
estimation, likelihoods, and a whole gamut of dynamic relationships
between random variables in joint probability distributions reassert
themselves amidst a population of models. The data may not be sampled,
but models moving through the high-dimensional vector spaces opened up
by having `all' the data transform it probabilistically. While not all
machine learners are strictly speaking probabilistic models,\footnote{Machine
  learning textbooks written by computer scientists tend to define
  probabilistic models more narrowly. As Peter Flach suggests:

  \begin{quote}
  Probabilistic models view learning as a process of reducing
  uncertainty using data. For instance, a Bayesian classifier models the
  posterior distribution \(P(Y|X)\) (or its counterpart, the likelihood
  function \(P(X|Y)\)) which tells me the class distribution \(Y\) after
  observing the features values \(X\) \autocite[47]{Flach_2012}
  \end{quote}

  But whether they are probabilistic in this sense or not, the
  evaluation and configuring of machine learners irreducibly depends on
  a statistical treatment of errors and their trade-offs.
  \index{Flach, Peter}} machine learners relate to themselves and the
data as populations defined by probability distributions.

Machine learning inhabits a reality that had already introjected
statistical realities at least a century earlier, whether through the
social physics of Quetelet, the biopolitical norms of Francis Galton and
his regression to the mean (the linear model of regression is probably
the basic machine learning model) or later, in the probability functions
of quantum mechanics in early twentieth century physics.
\index{Galton, Francis!regression to mean}
\index{Quetelet, Adolphe!social physics}
\index{probabilisation!quantum mechanics} Assembling an aggregate
reality of many devices, machine learning inverts probability
distributions. In this inversion, probability distributions, which had
become the operational statement and model of truth for many different
kinds of populations, fold back or re-distribute themselves into devices
such as machine learners whose variations and uncertainties become
populations. Populations of models are sampled, measured, and aggregated
in the ongoing production of statistical realities whose object is no
longer a property of individual members of a population (their height,
their life-expectancy, their chance of HIV/AIDS), but a population of
models of populations.
