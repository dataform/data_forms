\documentclass[../book.tex]{subfiles}

\chapter{Conclusion: Out of the Data}
\label{ch:conclusion}

\begin{quote}
\texttt{These\ diagrams\ of\ the\ diagrammatic\ domains,\ \ they\ kernel\ together\ in\ localization.}
\end{quote}

\begin{quote}
\texttt{In\ this\ contrusion\ of\ major\ forms\ of\ invention\ in\ natures\ in\ machine\ learning\ techniques,\ inter-places,\ leveraged\ in\ and\ distributed.}
\end{quote}

The two sentences above are the products of a generative model
\index{model!generative} trained on the raw text of this book. Without
any model of syntax, any dictionary of words or terms, relying purely on
character sequences as probability distributions, the neural network
that sampled these sentences out of its own unsupervised model of the
book vectorised as data was primed with starting text of `\texttt{If}.'
\index{machine learner!neural network} `Diagrams of the diagrammatic
domains,' kernelling together in localization, a `contrusion' of major
forms of invention in natures, in machine learning techniques, leveraged
in and distributed in inter-places: all of that has been put quite well
by the generative model, a two-layer `long short term memory' recurrent
neural net \autocite{Karpathy_2016}.

I began with a relatively limited question: if machine learning is
transforming the production of knowledge, might the practice of critical
thought itself change, whether in its empirical or theoretical
orientations? Could the `experimentation of concepts'
\autocite[153]{Stengers_2000} work with machine learning? My answer is
provisionally affirmative. If a book could be a generative model, then I
hope this auto-archaeology \index{archaeology!auto} might generate or
multiply the capacity to problematize the present. For such a machine
learner, a model that would learn machine learning in order to diagram a
diagrammatic domain, predictions would figure less as statements that
rank, order and classify, than as a technology of critical
experimentation, a means of effecting a certain number of transformative
operations on one's own conduct, thinking and ways of being amidst the
determinations of contemporary reality. It would function as a mode of
experimentation on statements. \index{statements!experimentation on}

\section{250,000 machine learners}\label{machine-learners}

For at least 230,800 human machine learners -- the number of unique
authors listed in the corpus of machine learning research literature I
have been drawing on -- \index{machine learner!number of}, a new kind of
operational formation jells in machine learning. People and things,
knowledge and power, combine in novel forms to generate statements.
Understanding the distribution and production of elements that make up
this emerging common space of decision, classification, prediction and
anticipation matters contemporary critical thought in its engagement
with power, production, conduct, communication, ways of being and
thinking, materiality and experience.

Let us take 146,000 scientific articles, publications and books as
statements concerning operations occurring in a variety of sites, modes,
and settings connected in the operational formation we are
discussing.\index{operational formation!constrast with discursive formation}.
As in Foucault's discursive formations, statements in operational
formations function by reference to the position of a subject
(\index{subject!position of} the expert, the engineer, the doctor, the
patient, the judge, the teacher, the student), amidst an organised or
grouped accumulation of devices, settings and fields
(positivity\index{positivity}), and with greater or lesser reference to
the practices of human-machine interaction. For instance, writing the
code that allows the recurrent neural net to build a generative model of
this text.

Although subjects for Foucault do not author statements, the assignment
of subject positions always passes through a human subject. In
operational formations, subject positions are less distinct, yet highly
populated (as the 230,000 authors of these paper suggest). The
machine-human mixing in operational formations is highly variable,
dynamic and mutable, sometimes planing through code, sometimes
diagrammed in visible forms such as graphs and tables, and often
ramifying through infrastructures. \index{statements!human-machine}

Affective elements have a long-standing connection with computation.
Elizabeth Wilson's study, \emph{Affect and Artificial Intelligence}
\autocite{Wilson_2010}, draws on a combination of psychoanalytic,
psychological and archival materials discussing the work of key figures
in the early history of artificial intelligence such as Alan Turing on
intelligent machinery, Warren McCulloch and Walter Pitts on neural nets,
and recent examples of affective computing and robots such as the MIT
robot Kismet.
\index{Wilson, Elizabeth!on artificial intelligence and affect} Her
framing of the psychic nexus with machines such as the perceptron is
provocative:

\begin{quote}
Sometimes machines are the very means by which we can stay alive
psychically, and they can just as readily be a means for affective
expansion and amplification as for affective attenuation. This is
especially the case of computational machines
\autocite[30]{Wilson_2010}.
\end{quote}

Under what conditions do machines and for present purposes,
computational machines, become `the very means we can stay alive
psychically'? Wilson addresses this question by positing `some kind of
intrinsic affinity, some kind of intuitive alliance between the machinic
and the affective, between calculation and feeling' (31), and suggesting
that the `one of the most important challenges will be to operationalize
affectivity in ways that facilitate pathways of introjection between
humans and machines' (31). \index{operational formation!affect in}
\index{artificial intelligence!affect in} Introjection, the process of
bringing the world within self is, according to psychoanalytic accounts
of subjectivity, crucial to the formation of `a stable subject position'
(25). Wilson envisages introjection of machine processes as a good, not
as a failure or attenuation of relation to the world.

While I tend to go in the same direction as Wilson in relation to
`affective expansion', I don't see that expansion as unfolding from
introjection, but rather from an intensification of diagrammatic
processes, the act of creating a `concrete being, an intersecting of
references' or abstraction \autocite[85]{Stengers_2000}
\index {diagram!affect of}

\section{A summary of the argument}\label{a-summary-of-the-argument}

I have been experimenting with abstraction in midst of data practices of
machine learning. Let me resume the argument of the book, an
archaeological argument that excavates seven major facets or
intersecting planes that belong to the machine learning as an
operational formation. \index{operational formation} Chapter
\ref{ch:diagram} addressed the problem of where amidst the mire of data,
mathematics, code, infrastructures, scientific and other knowledge
fields, a critical engagement with machine learning might situate
itself. I suggested that we should consider the formal, mathematical
abstraction and certain transformations in the production of software
associated with machine learning as diagrammatic processes that organise
and assemble human-machine relations. \index{human-machine relations}
Amidst a great accumulation of statements, figures, techniques,
constructs, datasets and code implementations derived from many
settings, the task is to map the intersecting references, the diagonal
connections, and the transformatinos and substitutions that weave
through machine learning. The positivity \index{positivity} of machine
learning, its specific forms of accumulation, regularity and rarety do
not attest to the power of algorithms but rather lend liveliness to the
field by concentrating expressions from many regions.

Chapter \ref{ch:vector} examined the practices of vectorising data,
situating machine learners themselves in an organised, dimensioned space
accommodating an increasing repertoire of transformations operating on
vectors. \index{vectorisation} Viewed as another mutation of the tabular
grid, vector space invites transformations of data. Machine learning is
a practice of working with data to accommodate all differences within an
expanding dimensional space, a space in which data is under the strain
of smooth surfaces, straight lines, regular curves and hyper-planes.
\index{data!strain} Both in terms of infrastructure and epistemic
cultures, the vector space abstracts and concretises
\index{abstraction!see concretisation} spaces inside data.

What is learning in machine learning? If information and computation can
be understood as responding to a crisis in control, what do machine
learners do? Chapter \ref{ch:function} examined how learning institutes
experimental relays between operation and observation in optimising
functions that predict and classify. The proliferation of methods and
devices in machine learning and the attempts to unify them as `learners'
was understood as a result of this entwining of operations and
observations. The interplay between operational transformations and
observational functions in optimisation accounts for much of the
`learning' effect in machine learning. \index{learning!optimisation}

An important and wide-reaching critical strand of work in humanities and
social sciences over the last few decades has focused on knowledge in
its entanglements with apparatuses of governmentalised power.
Populations and other large aggregates have been central objects of
concern.\index{population!power relations in} They remain so in
contemporary operational formations, although under somewhat altered
conditions. Having all the data, chapter \ref{ch:probability} suggested,
is not the principal stake in contemporary data cultures. Instead, the
probabilisation \index{probabilisation} of both data and machine
learners as populations, as distributed probabilities, indicates a
different axis along which power-knowledge develops in machine learning.

What happens to differences amidst vectorisation, learning as
optimisation, probabilisation and the generalized diagrammatic
abstraction of machine learning? \index{differences}. Are all
differences reduced to quantitative comparisons? Treated as pattern,
chapter \ref{ch:pattern} explored different treatments of difference in
machine learning. Differences bifurcate between infinitesimal graduation
and rigid decision boundaries, sometimes blurring or overlapping, and
sometimes distributed into inaccessibly high-dimensional inner data
spaces. The archaeological task amidst the dispersed patterns is to
locate differences in kind.

Rather than any new materiality, I have pointed to transformations in
referentiality associated with machine learning. From the standpoint of
operational \gls{archaeology}, the materiality of machine learning
refers to the practices of re-use that stabilise references. Science, by
virtue of its experimental inventiveness and truth-authority,
cross-validates the referentiality of machine learning.
\index{science!referentiality of} The topic of chapter \ref{ch:genome}
was a particularly data-intensive contemporary scientific hyperobject,
the genome. As a data form, genomic sequence data provokes re-use,
transcription and transmission of classifications and predictions. This
incites both infrastructural transformations but also new
concretisations of the hyperobject (as for instance in genome wide
association studies).

Finally, chapter \ref{ch:subject} explored the subject position of
machine learners. Within operational formations, subject positions arise
in gaps between operations and statements concerning operations. The
argument here concerned human-machine differences and the dispersion of
subject positions through operations that alter those differences. Even
amongst machine learners themselves, subject positions are not fixed or
unified. The deep neural networks that beat Go champions in 2015 and
2016 \autocite{Silver_2016} or developed hitherto unseen tactics in
playing Atari computer games \autocite{Mnih_2015} evidence the deeply
competitive or test-based administration of this gap.
\index{machine learning!competition in}

\section{In-situ hybridization}\label{in-situ-hybridization}

Beyond these facets of the argument concerning abstraction, inclusion,
control, multiplicity, differences, materiality and subject positions,
another argument shaped discussion in the preceding chapters, one that
affectively underpins of the writing. A central problem for critical
thought today (and by critical thought I mean post-Foucaultean
engagements with the events that constitute us subjects of what we say,
do and think \index{critical thought}) concerns how to engage with
operational formations. To an even greater extant than the discursive
formations that Foucault and many subsequent scholars have analysed,
operational formations in production, communication, and the regulation
of conduct become the field in which the work of ethics and politics
takes place.
\index{operational formation!compared to discursive formation}

The problem of engagement with operational formations is not so much how
to gain control, or challenge the asymmetries of access and control that
loom so large in them (Facebook can machine learn exponentially more
patterns than I can), but to begin to grasp the forms of change that are
possible and desirable. Mark Hansen has, for instance, posed the
challenge of engaging with data-intensive prediction directly in terms
of experience. He writes:

\begin{quote}
this imperative enjoins us to use the technologies of data capture,
analysis and prediction to create a feed-forward structure capable of
marshaling the full productive potentiality of data -- its commonality,
accessibility, and openness -- in order to improve, indeed to improve by
\emph{intensifying}, our experience \autocite[77]{Hansen_2015}
\index{Hansen, Mark!using potentiality of data}
\end{quote}

Treating prediction as more than means of disciplinary control, and
instead as a resource for individuals and collective to modulate
experience, Hansen's project draws on an extensive engagement with
phenomenology and Whitehead's philosophy. The crucial task in his view
is creative or inventive: the `feed-forward structure' must marshal `the
productive potentiality of data.'

One way to do this is broadly aligned with Foucault's emphasis in his
later work on care of the self. Technologies of the self `permit
individuals to effect a certain number of operations of their bodies and
social, thoughts, conduct and ways of being, so as to transform
themselves in order to attain a certain state of happiness, purity,
wisdom, perfection or even immortality' \autocite[225]{Foucault_1997}.
Could Hansen's feed-forward structure -- the term itself referring to
the first phase of neural net's learning -- operate as a technology of
the self, not so much focused on improvement or perfection of experience
but in name of the potential to invent new tests of and new relations to
pressing realities? For scholars producing critical knowledge in
humanities and social science through a variety of textual, empirical,
theoretical and increasingly implicitly or explicitly computational
practices, technologies of self offer a concrete path wending a way into
domains of production, communication and governance. Rather than
immortality or purity, operations effected on ways of thinking, living
and being might transform oneself in the interests of a limited
experience of freedom. \index{technologies of self!machine learning as}

Under what conditions could something like care of the self and
technologies of the self have any purchase, relevance or even toehold in
the operational formation of machine learning? Five elements, it seems
to me, need to be assembled in order to think through that conjunction.
The recognition of ourselves as subjects of machine learning is an
elementary archaeological task. Whether in relation to knowledge,
communication (in the broadest sense), conduct or ways of living, this
recognition relies on a description of practices associated with
differences, multiplicities, materialities, knowledges and control.
Second, as I have endeavoured to emphasise in describing machine
learning as an operational formation, the liveliness of machine learning
should be understood as a localisation of power-knowledge relations, or
a primary field of expressions issuing from many parts (to paraphrase
Whitehead \index{Whitehead, Alfred North!life}). `They kernel together
in localization' as my recurrent neural network
\index{machine learner!neural network!recurrent} puts it. Third, while
the accumulating plethora of techniques, applications and sites is
neither unified by a master algorithm or by a latent, underlying
meaning, it does demonstrate regularities and point of indetermination
or slippage. Fourth, understood as a field of the expression of many
parts, an operational formation can also be site of collective
individuation. Participating in a collective, individual subjects, far
from losing whatever defines their unique or essential identity, gain
the chance to individuate, at least in part, the share of pre-individual
reality that marks the collective within them. Fifth, by participating
in a collective, even an operational formation, individuals may
transform themselves (in order to attain certain states or experiences),
but also affect the collective itself.

Whether this might affect the internet filter bubble
\autocite{Pariser_2011}, the `stack to come' \autocite{Bratton_2016},
digital citizenship \autocite{Isin_2015}, the character of work
\autocite{Brynjolfsson_2014}, the fabric of experience
\autocite{Hansen_2015} or what counts as knowledge
\autocite{Bowker_2014} is hard to say. As an operational formation,
machine learning does not determine anything in its operations, even if
it connects directly to strategies of power. Foucault writes that
`archaeology describes the different spaces of dissension'
\autocite[152]{Foucault_1972} \index{archaeology!spaces of dissension}.
These spaces of dissension, it seems to me, form a field in which
initiatives, individuations and technologies of the self might
articulate a certain number of transformative operations.

\section{Critical operational
practice?}\label{critical-operational-practice}

Under what conditions would that experimental practice and operation on
ways of thinking and saying be divergent rather than convergent? Writing
this book, and learning to machine learn in order to write about machine
learning, involves participation in a collective, the collective of at
least 230,000 scientist-machine learners, and the tends of thousands of
programmers developing machine learners evident on Github.com. By
participating in the collective operational formation, running the risk
of being mobilized by existing interests, we might also individuate
differently a share of the pre-individual reality included within us
\autocite[79]{Virno_2004}. \index{Virno, Paolo!collective individuation}
\index{collective!individuation of} Like Anne-Marie Mol's
`praxiography,' which seeks to maintain reality multiples in describing
practice \autocite[6]{Mol_2003}, the description of machine learning as
data practice intends to sustain the multiple of reality by identifying
the practices that make it multiple.
\index{Mol, Anne-Marie!on praxiography}
\index{data practice!as multiple}

The path I've taken here combines writing (a discursive practice) and
coding (an operational practice). Writing about machine learning is a
practice of diagrammatically mapping the re-iterative drawing of
human-machine relations in code, and in particular, in coding that
learns from data. Datasets, scientific and engineering publications,
textbooks such as \emph{Elements of Statistical Learning}, software
libraries and packages, spectacular demonstrations comprise a whole
series of criss-crossings. While not the path that everyone would or
should want to take, for me moving into the data like or as a machine
learner perhaps allows writing to become more diagrammatic. `Between the
figure and the text we must admit a whole series of criss-crossings'
wrote Foucault \autocite[66]{Foucault_1972}, in defining
\gls{archaeology} as a mode of exploration of knowledges, politics and
ways of being.

Very mundanely, I've read articles and books, downloaded data and
software libraries, watched Youtube lectures and presentations,
configured and written bits of code and text, made plots and diagrams,
and done much configuration work across various platforms (Github.com,
linux, Google Compute, \texttt{R}, \texttt{python} and
\texttt{ipython}). Amidst all of this data practice (and much
practising), there is no reason to assume that learning machine learning
is solely the performance of a conscious subject. When we look at an
equation repeatedly, when we comply with the machine learning injunction
to `find a useful approximation \(\hat(f)(x)\) to the function \(f(x)\)
that underlies the predictive relationship between input and output'
\autocite[28]{Hastie_2009} by writing code to cross-validate a model, we
surrender to `learning' that, however fascinating or surprising, is not
that of a conscious human subject but also of human-machine assemblage.
To the extent that it is archaeological, operational, diagrammatic
writing vibrates around the axis of knowledge/practice, not
knowledge/consciousness. \index{archaeology!writing practice}

\section{Obstacles to the work of freeing machine
learning}\label{obstacles-to-the-work-of-freeing-machine-learning}

As I have emphasised on several occasions, machine learning is an uneasy
mixture of massively repeated and familiar forms, and something that is
not easily understood. On the one hand, the level of imitation,
duplications, copying and reproduction associated with the techniques
suggests that a process of remaking the world according to particular
forms is in process (for instance, in chapter \ref{ch:probability} we
saw how Naive Bayes classifiers are almost demonstrated on spam
classification problems.) The scientific and engineering literature,
with its really frequent variations on similar themes, suggests that
imitation and copying are very much at the heart of the movements I have
been describing. This is nothing new.
\index{machine learning!imitation in} It would be strange of these
techniques were not subject to imitation and emulation. That imitation
is predictable. We expect it and can account for it
sociologically.\footnote{Accounts that might do this can be found in
  science and technology studies, particularly in actor-network theory
  versions, as well as in recent social and cultural theory that, for
  instance, draws on the work of the 19th century French sociologist,
  Gabriele Tarde \autocites{Tarde_1902}{Borch_2005}.} Some symptoms of
these imitative fluxes can be found in the scientific and engineering
literature. As we have seen, work on image and video classification, on
text and speech, on gene interaction prediction or above all, on
predictions of relations or associations between people and things
(usually commodities, but not always) is striking in its persevering
homogeneity. Moreover, the powerful aspirations evident amongst large
media platforms such as Baidu, Google and Facebook to re-ground machine
learning in the project of artificial intelligence amidst social media
or web page-related data in many ways continues business as usual for
computer scientists \autocite{Gulcehre_2014}.

How would we get any sense of what is not so easily digested and laid
out in social practice? Archaeologies of operational formations aim to
present some of the necessary elements for that purpose. In the closing
pages of \emph{The Archaeology of Knowledge}, Foucault writes:

\begin{quote}
the positivities that I have tried to establish must not be understood
as a set of determinations imposed from the outside on the thought of
individuals, or inhabiting it from the inside, in advance as it were;
they constitute rather the set of conditions in accordance with which a
practice is exercised, in accordance with which that practices gives
rise to partially or totally new statements, and in accordance with
which it can be modified. These positivities are no so much limitations
imposed on the initiative of subjects as the field in which that
initiative is articulated \autocite[208-209]{Foucault_1972}.
\index{positivity}
\end{quote}

Here Foucault refers to the restricted freedom that discursive practices
and formations open for us. If it is increasingly difficult for science,
media, government and business to think and act outside data. And yet
Foucault is quite clear that amidst the positivities of knowledge
production, knowing the conditions, setting out the rules, and
identifying the relations that striate the density and complexity of
practice is a pre-condition to any transformations in practice.

As a data practice, however, machine learning is not entirely
predictable. Machine learners, as we have seen, vary too much, they are
biased, they overfit, they underfit, and they often fail to generalise.
\index{machine learning!limitations of} Despite this, they have enormous
allure. In the history of automata, automation and animation, kinetic
lures have long exercised fascination, and this may be part of the
effect of machine learning. Animating transformations of data (think of
the 366 times the logistic regression traverses the
\texttt{South\ African\ Heart\ Disease} dataset), and then looking at
those optimising animations as `learning' generates operational power
dynamics. \index{automation!animation of}

Machine learning more broadly attracts infrastructural, technical,
professional, semiotic and financial diagonals -- think of the upswing
in Google searches for `machine learning' shown in figure
\ref{fig:google_trends} in chapter \ref{ch:introduction} -- that render
its traits more real, more thickly transformative and more `performant.'
Yet such performant diagrams generate referential effects. Machine
learning becomes ontologically potent. As Maurizio Lazzarato writes in
\emph{Signs and Machines}, `ontological mutations are always machinic.
They are never the simple result of the actions or choices of the
``man'' who, leaving the assemblage, removes himself from the non-human,
technical, or incorporeal elements that constitute him'
\autocite[83]{Lazzarato_2014}.
\index{Lazzarato, Maurizio!asemiotic machine}

New machine learners arise from diagrammatic superimposition of existing
practices or procedures. Neural networks are like a massively
proliferating nest of perceptrons. Moreover, machine learning techniques
often repeat something familiar by very different means (think of how
\texttt{kittydar} treats photographs, or how a decision tree is legible
but often unfamiliar). The event, then, resides less in either something
intrinsic to devices operating as algorithmic models, or in something
about the domains and places in which the devices operate (biomedicine,
state security and intelligence agencies, finance, business, commerce,
science, etc.). Perhaps it is a rather more modest event in which the
tending of abstractions through estimation, optimisation,
high-dimensional vectorisation, probabilistic mixing of latent and
feature variables, and imputation unevenly replace existing ontological
and epistemic norms of verification, objectification, and attribution.
\index{machine learning!unpredictable operation of}

I have been less interested in treating these techniques as the
predictable re-animation of alienated reason, and more inclined to look
for those elements in machine learning that diagrammatically abstract
away from structures of representations, subjectification or indeed
implementation associated with platforms, services and products (for
instance, the interminable implementations of document classifiers,
sentiment analyses, or image labelling, or handwritten digit
recognition, or autonomous navigation, etc.).
