\documentclass[book.tex]{subfiles}

\documentclass[book.tex]{subfiles}

\documentclass[book.tex]{subfiles}

\chapter{Acknowledgments}
\label{ch:acknowledgments}

From 2007-2012, I benefited greatly from a research position in the UK
Economic and Social Research Council-funded Centre for Economic and
Social Aspects of Genomics at Lancaster University. Certain colleagues
there, initially in the `Sociomics Core Facility', participated in the
inception of this book. Ruth McNally with her almost geeky interest in
genomics, Paul Oldham with his enthusiasm for `all the data,' Maureen
McNeil with her critical acuity, and Brian Wynne with his connective
thought were participants in many discussions concerning the
transformation of life sciences around which my interest in machine
learning first crystallised.

The Technology in Practice Group at ITU Copenhagen hosted some of the
research work during 2014. Brit Ross Winthereik in particular made it
possible for to develop some of the key ideas. I was lucky too to be
part of an excellent research team on `Socialising Big Data'(2013-2015)
that included Penny Harvey, Celia Lury, Ruth McNally and Evelyn Ruppert.
We had excellent discussions.

My colleagues in the science studies at Lancaster, especially Maggie
Mort, Lucy Suchman, and Claire Waterton are always a delight to work
with. I have also been fortunate to have worked with inspiring and
adventurous doctoral students at Lancaster during the writing of this
book. Lara Houston, Mette Kragh Furbo, Felipe Raglianti, Emils Kilis,
Xaroula Charalampia, and Nina Ellis have all helped and provided
inspiration in different ways. Sjoerd Bollebakker very kindly updated
many of the scientific literature searches towards the end of the book's
writing.

Various academic staff in the Department of Applied Mathematics and
Statistics at Lancaster University shepherded me through post-graduate
statistics training courses: Brian Francis for his course of `Data
Mining,' David Lucy for his course on `Bayesian Statistics', Thomas Jakl
for his course `Genomic Data Analysis' and TBA's course on `Missing
Data.' Since 2015, I've also come to know some machine learners much
better through the Data Science Institute, Lancaster University.

My inestimable friends in the \emph{Computational Cultures} editorial
group have listened to and irrationally encouraged various threads of
work runnning the book. I warmly acknowledge the existence of Matthew
Fuller, Andrew Goffey, Graham Harwood and Olga Goriunova. Nina Wakeford
at once point encouraged me to be naïve.

Far away in Australia, Anna Munster has stayed in touch. As always,
Celia Roberts helped me sort out what I really want to do.
